\documentclass[11pt]{article}
\usepackage[margin=1in]{geometry}
\usepackage[T1]{fontenc}
\usepackage[polish]{babel}
\usepackage[utf8]{inputenc}
\usepackage{lmodern}
\usepackage{hyperref}
\usepackage{enumitem}

\selectlanguage{polish}

\begin{document}

\title{Zasady krytycznego myślenia - zadanie 2}
\author{Wojciech Kołowski}
\date{10 czerwca 2018}
\maketitle

\par W ramach zadania przeanalizuję tekst zatytułowany ``Bezwarunkowy dochód podstawowy: Czy się stoi, czy się leży, godne życie się należy''. Jest to wywiad przeprowadzony przez Karolinę Zbytniewską z prof. Guyem Standingiem, prezesem Basic Income Earth Network. Tekst dostępny jest tu:

\url{https://www.euractiv.pl/section/demokracja/news/bezwarunkowy-dochod-podstawowy-sie-stoi-sie-lezy-godne-zycie-sie-nalezy/} \\

\par Zacznijmy od sporządzenia konspektu. Jako że tekst ma charakter wywiadu, spróbujemy streścić każdą wypowiedź do jednego zdania zawierającego całą jej myśl. Taki konspekt następnie ściśniemy do postaci listy argumentów za dochodem podstawowym i płynących z nich wniosków, której przyjrzymy się nieco bliżej. \\

\par Objaśnienia: KZ — Karolina Zbytniewska, GS — Guy Standing, BDP — bezwarunkowy dochód podstawowy.
\par GS: 99\% ludzi chce poprawić swoje życie i dalej będzie chciało po wprowadzeniu BDP.
\par KZ: Niewiele państw stać na BDP.
\par GS: Argumenty moralne i filozoficzne za BDP są ważniejsze niż argumenty ekonomiczne. Chodzi o likwidację biedy i zapewnienie sprawiedliwości społecznej.
\par KZ: W Zachodnich krajach jest sprawiedliwość społeczna, tj. równe szanse. Możliwa jest ścieżka ``od zera do milionera''.
\par GS: Bogactwo i zarobki zawdzięczamy rodzicom.
\par KZ: Ludzie nie rodzą się równi. Wiele zależy od czynników, na które nie mamy wpływu.
\par GS: Bogaci ludzie bogacą się na zanieczyszczaniu środowiska. BDP to sprawiedliwość ekologiczna. BDP to wolność od poniżającej pracy, chamskiego szefa, biurokracji.
\par KZ: Zasiłki z urzędu pracy to jałmużna, BDP nie.
\par GS: BDP zwiększa potencjał ludzki i kapitał społeczny.
\par KZ: BDP jest testowany w Finlandii.
\par GS: Eksperyment fiński to nie BDP, ale wyniki i tak będą pozytywne.
\par KZ: Warunkowy dochód podstawowy jest niesprawiedliwy.
\par GS: Bezwarunkowość dochodu podstawowego jest bardzo ważna.
\par KZ: Być może nie każde społeczeństwo jest gotowe na BDP. Polska na pewno nie, bo żule piją wino pod sklepem.
\par GS: Ludzie, których stać na podstawowe wydatki, są bardziej odpowiedzialni i wydają mniej na używki. Żule piją pod sklepem, bo pomoc społeczna poniosła porażkę.
\par KZ: 500+ zmniejszyło zatrudnienie wśród słabo wykształconych i kobiet w średnim wieku.
\par GS: 500+ promuje patriarchat, paternalizm oraz katolicyzm. Rzucanie ``niegodnej'' pracy jest słuszne. BDP może prowadzić do poprawy warunków zatrudnienia.
\par KZ: Sprawiedliwość społeczna jest ważniejsza od pieniędzy.
\par GS: Wdrożenie BDP to tylko kwestia mądrej strategii. Każdy ma prawo do minimum socjalnego, które ma zapewnić mu państwo.
\par KZ: Jak finansować BDP?
\par GS: Narodowe fundusze kapitałowe, jak w Norwegii czy na Alasce. Polska wydała 100 mld euro na subsydiowanie bogatych, a mogłaby na biednych. EBC uratował banki kwotą 1 bln euro, a mógł rozdać te pieniądze biednym.
\par KZ: Skoro BDP jest takie dobre i państwa stać na nie, to dlaczego nie jest wdrażane?
\par GS: Ekonomiści służą bankom i korporacjom. \\

\par W tak sporządzonym konspekcie widzimy, że przez wywiad przewijają się kilkukrotnie podobne wątki — kwestia ludzi bogatych, sprawiedliwość społeczna, finansowanie BDP etc. Spróbujmy sensownie uporządkować postulaty Guya Standinga, abyśmy mogli łatwiej je przeanalizować. Kilka wątków możemy pominąć: wątpliwości wyrażane przez prowadzącą wywiad (musiała udawać adwokata diabła), historyjkę o żulach pijących pod sklepem, skutki wprowadzenia 500+ oraz płytkie rozważania o różnych miejscach, gdzie testowane jest BDP. Zrekonstruowane rozumowanie prof. Standinga na temat BDP wygląda tak:

\begin{enumerate}
  \item Ludzie chcą poprawić swoje życie.
  \item BDP nie sprawi, że ludzie przestaną chcieć poprawić swoje życie.
  \item Ludzie nie są równi. Bogactwo lub bieda zależą od rodziców.
  \item Każdy ma prawo do minimum socjalnego.
  \item BDP zlikwiduje biedę i wprowadzi sprawiedliwość społeczną.
  \item Obecny system zabezpieczeń społecznych jest nieskuteczny.
  \item Bezwarunkowość jest w BDP bardzo istotna.
  \item BDP zapewnia godne warunki pracy.
  \item BDP zwiększa kapitał społeczny.
  \item Ludzie są gotowi na BDP.
  \item Bogaci ludzie zanieczyszczają środowisko.
  \item BDP zrekompensuje biednym niszczenie środowiska.
  \item BDP można finansować z pieniędzy unijnych, zamiast ratowania banków, z funduszy w stylu norweskim lub alaskim.
  \item Wdrożenie BDP jest możliwe.
  \item Ekonomiści nie są przychylni BDP, bo są sługusami banków i korporacji.
  \item Należy wprowadzić BDP.
\end{enumerate}

\par Przyjrzyjmy się po krótce strukturze rozumowania: w punkcie (1) Standing stiwerdza pewien fakt na temat ludzkiej natury, zaś w (2) stwierdza bez argumentowania, że BDP nie zmienia prawdziwości (1). (3) to stwierdzenie pewnego faktu statystycznego, zaś (4) to postulat polityczny, filozoficzny i moralny powiązany z (3). W punkcie (5) Standing stwierdza, że BDP realizuje postulat (4). (6) to diagnoza obecnego systemu zabezpieczeń społecznych odnosząca się do jego warunkowości i ``paternalizmu''. W (7) Standing stwierdza, że BDP nie posiada takich wad jak obecny system. (8) i (9) to twierdzenia na temat społecznych konsekwencji BDP, zaś (10) to wyciągnięty z nich wniosek. (11) zwraca uwagę na rzekomy problem, a (12) wskazuje jego rozwiązanie. (13) opisuje sposób finansowania BDP. (14) to wniosek płynący z (13). (15) to opinia na temat przyczyn marnych postępów we wprowadzaniu BDP. (16) to główny wniosek rozumowania, wynikający z poprzednich, pomniejszych wniosków, który pośrednio popierają wszystkie przesłanki: należy wprowadzić BDP.

\par Przejdźmy do analizy możliwości by przekonać się, czy rozumowanie jest poprawne:

\begin{enumerate}
  \item \textbf{Ludzie chcą poprawić swoje życie}. Standing twierdzi, że fakt ten dotyczy 99\% ludzi, jest to jednak niedoszacowanie. Jest to postulat niemal metafizyczny, lecz każdy z nas jest w stanie za pomocą introspekcji przekonać się o jego absolutnej słuszności. Każdy człowiek, który działa (a więc niebędący w śpiączce, martwy, zahibernowany etc.) robi to, co uważa za korzystne dla siebie, czyli to, co prowadzi do poprawy swojego życia.
  \item \textbf{BDP nie przeszkadza w poprawianiu swojego życia}. W pewnym sensie Standing ma rację — wprowadzenie takiej czy innej ustawy nie jest w stanie zmienić fundamentalnej natury człowieka. Jednak sformułowanie, jakiego używa (``przestaną chcieć'') sugeruje, że rozważa on jedynie ludzi żyjących w chwili wprowadzenia BDP (a więc także najprawdopodobniej dorosłych). Jest to olbrzymie przeoczenie, gdyż nie wiadomo, jaki wpływ tak gigantyczna zmiana będzie miała na przyszłe pokolenia i ich podejście do życia. Nie znaczy to bynajmniej, że natura ludzka uległaby zmianie — w krajach komunistycznych ludzie również pragnęli poprawiać swoje życie, ale jakoś niezbyt im się to udawało. Tutaj mogłoby być podobnie — ludzie od urodzenia przyzwyczajeni do tego, że pieniądze dostaje się za nic od rządu, mogliby nie być zbyt skłonni do produktywnej pracy.
  \item \textbf{Ludzie nie są równi. Bogactwo lub bieda zależą od rodziców}. Standing znów ma rację, ale chyba w innym sensie niż zamierzał. Faktycznie bogactwo i bieda zależą głównie od rodziców, ale nie od ich majątku, tylko od ich genów. Ze statystyk (których źródła nie przytoczę) wynika, że jedynie ok. 25\% amerykańskich fortun pochodzi z dziedziczenia — pozostałych ludzie dorobili się własną pracą i przedsiębiorczością. Człowiek posiadający IQ kwalifikujące go do górnych 5\% ma dużo większą szansę zostać bogatym niż człowiek urodzony w rodzinie należącej do górnych 5\% pod względem bogactwa.
  \item \textbf{Każdy ma prawo do minimum socjalnego}. Absolutnie nie zgadzam się z prof. Standingiem. Występuje tutaj pewna dualność, jakże łatwa do przeoczenia: prawa jednej osoby są obowiązkami drugiej. Jeżeli komuś należy się mieszkanie, to ktoś inny (państwo) musi mu je zapewnić. Państwo nie ma jednak żadnych mieszkań, ani nie ma niczego poza monopolem na przemoc. Pozostają mu więc dwa rozwiązania:
  \begin{itemize}
    \item ukraść komuś mieszkanie i dać potrzebującemu — jest to skrajnie nieakceptowalne i szybko doprowadziłoby do katastrofy
    \item ukraść komuś pieniądze (czyli opodatkować) i z tych pieniędzy zbudować/kupić/opłacić potrzebującemu mieszkanie
  \end{itemize}
  \par Oba rozwiązania sprowadzają się do użycia przemocy wobec uczciwych i produktywnych obywateli, mającego na celu pozbawienie ich majątku i przekazanie go obywatelom co najmniej mało produktywnym. Prawo do minimum socjalnego jest więc także obowiązkiem stosowania przemocy i kradzieży. Jeżeli ktoś uzurpuje sobie prawo do minimum socjalnego, to tak naprawdę uzurpuje sobie prawo do okradania innych.
  \item \textbf{BDP zlikwiduje biedę i wprowadzi sprawiedliwość społeczną}. Zdanie to jest wielką kumulacją niejasności. Mamy tu aż trzy niejasne pojęcia: BDP, bieda i sprawiedliwość społeczna.
  \begin{enumerate}[label=(\alph*)]
    \item Pojęcie BDP jest niejasne, gdyż tak naprawdę nie wiadomo, jak dochód podstawowy miałby zdaniem prof. Standinga wyglądać (przynajmniej na podstawie tylko tego tekstu). Jedyną jego cechą, na którą położono nacisk, jest bezwarunkowość, jednak i ona jest niejasna: czy BDP otrzymywałyby dzieci czy tylko dorośli? Imigranci, rezydenci czy tylko obywatele (jeżeli tak, to jakie są warunki zdobycia obywatelstwa)? Inne cechy, bez których nie sposób zwerfikować tego stwierdzenia, to np. kwota — gdyby była za niska, program by nie wypalił.
    \item Bieda jest pojęciem o sporym polu do nadużyć. Dwa podstawowe jej rodzaje to bieda relatywna i absolutna. Tej pierwszej z pewnością nie da się zwalczyć — jest niemożliwe, by wszyscy żyli na poziomie ``powyżej średniej'' (ani powyżej żadnego innego progu — zawsze ktoś będzie niżej, a ktoś inny wyżej). Jeżeli chodzi o biedę absolutną, to rodzi się pytanie, czy jest tu z czym walczyć, połączone z wątpliwościami co do sposobu określenia stanu biedności: biedak w Kalifornii wyleguje się pod palmami na słonecznej plaży, je darmowe posiłki i korzysta na swoim smartfonie z darmowego Internetu. Biedak w Zimbabwe... cóż.
    \item W przypadku sprawiedliwości społecznej podstawową wątpliwością jest przymiotnik ``społeczna'' — czym różni się ona od zwykłej sprawiedliwości? Jeżeli niczym, to po co w ogóle takie pojęcie (nie wspominając o tym, że rozdawanie pieniędzy za darmo wielu ludziom nie wydaje się być sprawiedliwe). Jeżeli sprawiedliwość społeczna to coś innego niż sprawiedliwość, to może naszym priorytetem powinno być wprowadzanie zwykłej sprawiedliwości?
  \end{enumerate}
  \par Podsumowując: nie sposób sprawdzić, czy twierdzenie z tego punktu jest prawdziwe, gdyż pojęcia są niejasne.
  \item \textbf{Obecny system zabezpieczeń społecznych jest nieskuteczny}. Jeżeli przyjmiemy, że celem takiego systemu jest zapewnienie człowiekowi ``godnego życia'', to rzeczywiście tak jest, przynajmniej w Polsce. Zasiłki dla bezrobotnych są niskie, podobnie jak emerytury i renty. Co więcej, system jest oszukiwany — zasiłki pobierają ludzie pracujący na czarno, a faktycznym bezrobotnym często one nie przysługują. Renty nie są przyznawane osobom ewidentnie niezdolnym do pracy, bywają zaś wyłudzane. Podobnie emerytury ubeków są gigantyczne, a ich ofiar — marne. Co więcej, system jest mocno zbiurokratyzowany, co rodzi marnotrawstwo środków oraz czasu i pracy ludzi.
  \par Waga tego argumentu leży jednak gdzie indziej: implicite zakłada on, że system zabezpieczeń społecznych powinien istnieć i jego istnienie jest korzystne, a jedynym problemem jest kwestia skutecznego jego skonstruowania. Pominięty jest zatem wspomniany już fakt, że prawo do rzeczy materialnych (w tym przypadku do opieki społecznej, występującej głównie pod postacią pieniędzy) jest prawem do kradzieży.
  \par Istnienie systemu wcale nie musi być korzystne — obecnemu systemowi emerytalnemu zarzuca się, że jest powodem niskiej dzietności w krajach Zachodu, co jest niekorzystne. System emerytalny jest też przykładem na to, że być może system zabezpieczeń społecznych jest niemożliwy — wygenerowany przez niego spadek dzietności doprowadzi prędzej czy później do jego upadku. W przypadku BDP może być podobnie — gdyby przysługiwał on imigrantom, to mógłby wywołać olbrzymią imigrację (choćby z krajów trzeciego świata), która spowodowałaby bankructwo systemu.
  \item \textbf{Bezwarunkowość jest w BDP bardzo istotna}. Patrz punkt (5a).
  \item \textbf{BDP zapewnia godne warunki pracy}. Argument ten jest klasycznym przykładem krótkowzroczności rodem z ``Co widać i czego nie widać'' Bastiata. W krótkim okresie faktycznie jest tak, jak twierdzi Standing — gdy zwiększa się dochód człowieka, jest on w stanie zaspokoić więcej swoich potrzeb, więc niezaspokojone pozostają jedynie potrzeby mniej pilne niż poprzednio. Potrzeby te człowiek ten ceni sobie mniej (gdyby cenił je bardziej od innych, to właśnie je zaspokoiłby w pierwszej kolejności), więc jest skłonny podjąć mniejsze wysiłki i wyrzeczenia w celu ich zaspokojenia. To podnosi jego pozycję negocjacyjną względem pracodawcy, co w przełożeniu na poziom makro jest czynnikiem sprzyjającym wzrostowi płac (choć mogą wystąpić inne, równoważące go czynniki, jak np. imigracja).
  \par W długim okresie jest jednak więcej możliwości, których prof. Standing nie uwzględnia. Jedną z nich jest fakt, że rosnące koszty płacy sprawiają, że rośnie także stopa zwrotu z inwestycji w automatyzację. Jeżeli nastąpił wzrost pensji pracowników, który nie wynika ze wzrostu ich produktywności, to ceteris paribus spadają zyski pracodawcy. Ma więc on motywację do cięcia kosztów, a bardzo dobrym na to sposobem jest zastąpienie ludzi maszynami — mogą one pracować 24/7, nie trzeba odprowadzać na nie składek i nie dotyczą ich takie same zasady BHP jak ludzi. Na koniec zauważmy jeszcze, że problem ten dotyczy najbardziej pracowników słabo wykształconych, wykonujących niezbyt skomplikowane, powtarzalne (a zatem łatwo automatyzowalne) czynności, którzy z reguły zarabiają niewiele. Ostatecznie próba pomocy biednym (w tym biednym pracującym) może doprowadzić do pogorszenia ich sytuacji. Jest to typowy paradoks interwencjonizmi państwowego: zastosowane środki osiągają skutek przeciwny do zamierzonego.
  \item \textbf{BDP zwiększa kapitał społeczny}. Prof. Standing stwierdza, że badania wykazują, że niepewność redukuje możliwości intelektualne. Nie przytacza jednak źródła, co stawia tę wypowiedź pod znakiem zapytania. Twierdzi też, że lęk i stres uniemożliwiają podejmowanie racjonalnych decyzji, argumentując przez introspekcję. Moim zdaniem jest to wątpliwe.
  \par Z pewnością BDP mógłby zwiększać indywidualny potencjał ludzi — jest powszechnie wiadomo, że osoby w dzieciństwie chronicznie niedożywione mają IQ niższe nawet o kilkanaście punktów. Problem ten jednak dotyczy krajów trzeciego świata, a nie bogatych krajów Zachodu. Jest to trochę paradoksalne — kraje, które najbardziej stać na BDP, odniosłyby z niego najmniejsze korzyści.
  \par Kolejnym problemem z tym argumentem jest to, że występuje w nim niejasne pojęcie ``kapitału społecznego''? Nie jest ono powszechnie znane, a w tekście nie zostało objaśnione. Co więcej, pojęcie to jest podejrzane. Cytat za Wikipedią: \textit{Social capital is a form of (...) capital in which (...) market agents produce goods and services not mainly for themselves, but for a common good}. Taka definicja tego pojęcia jest sprzeczna z podstawami ekonomii, które głoszą, że człowiek działa dla własnego zysku (niekoniecznie materialnego), zaś warunkiem działalności rynkowej jest opłacalność przedsięwzięcia.
  \par Standing zdaje się również nie uwzględniać faktu, że ``kapitał społeczny'' to nie wszystko. Bardzo ważny jest także system motywacji oraz wzorce i wnioski jakie można wynieść z obowiązującego systemu społecznego. Osoby długo pobierające zasiłki często wpadają w marazm, uczą się bezradności. Zasiłki stają się jedynym ich źródłem utrzymania (co prowadzi do życia na granicy ubóstwa), a te niekorzystne wzorce przekazują swoim dzieciom. Jest niewykluczone, że BDP mogłoby mieć podobne skutki (o ile byłoby odpowiednio wysokie, by można było przeżyć; jednak przeciwna możliwość przeczy samej idei BDP).
  \item \textbf{Ludzie są gotowi na BDP}. To stwierdzenie jest wnioskiem z dwóch poprzednich, ale wobec nieuwzględnionych w tamtych przypadkach możliwości jest dość wątpliwe. Ciekawym empirycznym argumentem przeciw niemu jest referendum, które odbyło się jakiś czas temu w Szwajcarii. Obywatele tego kraju miażdżącą większością odrzucili w nim pomysł wprowadzenia BDP w kwocie ok. 2500 franków. Z drugiej strony, nietrudno wyobrazić sobie biednych ludzi podatnych na populizm, którzy chętnie zagłosowaliby za dostawaniem darmowych pieniędzy, nawet w Polsce. Jest to kolejny paradoks: społeczeństwo bogate, które najbardziej stać na wprowadzenie BDP, jest przeciwne tej idei, zaś wydaje się że inne, dużo biedniejsze społeczeństwa, których nie stać na ten pomysł, byłyby bardziej skłonne go wdrożyć.
  \item \textbf{Bogaci ludzie zanieczyszczają środowisko}. Ten argument zdaje się być biciem w chochoła. Wyobraźmy sobie typową fabrykę zanieczyszczającą środowisko. Właściciele, zarząd, rada nadzorcza i kierownictwo to wprawdzie raczej zamożni ludzie, ale to nie oni własnoręcznie wylewają ścieki do rzeki — robią to pracownicy, a więc raczej ludzie mniej zamożni. Co więcej, nie jest temu winna żadna chciwość ani inne negatywne cechy stereotypowo przypisywane ludziom bogatym — jest to jedynie konsekwencja opłacalności takich a nie innych działań. Należy pamiętać również, że sporo zanieczyszczeń, np. spaliny ze starych, mało ekologicznych samochodów, albo palenie śmieciami w piecach jest domeną ludzi biednych.
  \par Ten argument jest też próbą wprowadzenia do dyskusji typowych dla marksizmu podziałów klasowych, które można następnie wykorzystać odwołując się do liczniejszej z klas (czyli do biednych) poprzez granie na emocjach.
  \item \textbf{BDP zrekompensuje biednym niszczenie środowiska}. Ten punkt jest wnioskiem z poprzedniego i w obliczu wysuniętych wobec niego wątpliwości sam jest wątpliwy. Co więcej, prof. Standing w ogóle nie próbuje nawet wytłumaczyć, w jaki sposób BDP miałoby rekompensować biednym zanieczyszczanie środowiska. Czy opodatkowanie bogatych w celu jego sfinansowania miałoby taki proekologiczny efekt? Nawet gdyby tak właśnie było, to jest to zasługa samego opodatkowania i nie można bezpośrednio przypisywać jej BDP. Co więcej, mogłoby stać się coś zupełnie przeciwnego: wzrost dochodów ludzi biednych mógłby doprowadzić do wzrostu konsumpcji, a ta z kolei do wzrostu produkcji i w efekcie do wzrostu zanieczyszczenia środowiska.
  \item \textbf{BDP można finansować z pieniędzy unijnych, zamiast ratowania banków, z funduszy w stylu norweskim lub alaskim}.
  \par Rozważmy każdą z tych kwestii osobno:
  \begin{itemize}
    \item Postulat finansowania BDP z pieniędzy unijnych brzmi jak wyznanie wiary w finansowe perpetuum mobile. Pieniądze unijne wszakże nie biorą się znikąd — pochodzą one ze składek członkowskich (pomijam kreację euro przez EBC — patrz kolejny punkt). Jeżeli kraju nie stać na BDP, to konieczność wpłacenia składki członkowskiej do unijnego budżetu mu w tym nie pomoże. Takie finansowanie byłoby możliwe tylko gdyby z unii można było wyciągnąć więcej pieniędzy niż się tam wkłada. Jest to jednak możliwe tylko w przypadku pojedynczych (i to biedniejszych) krajów, a więc byłby to sposób pasożytniczy, polegający na wprowadzeniu BDP w jednym kraju kosztem innych krajów. Finansowanie BDP w ten sposób na skalę całej unii nie jest możliwe.
    \item Pieniądze na ratowanie banków zostały, o ile mi wiadomo, po prostu wykreowane przez EBC. Postulat finansowania BDP z tych pieniędzy jest więc faktycznie postulatem finansowania BDP z dodruku. Dodruk pieniądza nie jest jednak dobrym pomysłem — inflacja pozbawia ludzi oszczędzających majątku kosztem dłużników. Dodruk w kwotach potrzebnych do finansowania BDP mógłby zaś bez problemu wywołać hiperinflację i załamanie się całego kraju — niedawnym przykładem hipertinflacji wynikającej z dodruku jest Wenezuela. Taki obrót spraw rzecz jasna nie pomógłby biednym. Wprost przeciwnie — pogorszyłby on ich sytuację, podobnie zresztą jak sytuację wszystkich innych mieszkańców kraju.
    \item Równie nierealnym pomysłem jest finansowanie BDP z funduszy podobnych do tych, jakie mają Norwegia lub Alaska, gdyż źródłem tych funduszy jest wydobycie ropy naftowej. Widać więc, że taki sposób finansowania nie zadziała w przypadku państw pozbawionych surowców naturalnych.
  \end{itemize}
  \item \textbf{Wdrożenie BDP jest możliwe}. Punkt ten jest wnioskiem płynącym z poprzedniego punktu. Wszystkie zaproponowane w nim sposoby finansowania BDP po bliższych oględzinach okazują się nierealne, co mocno wskazuje na niesłuszność tego wniosku. Nim jednak go odrzucimy, spróbujmy własnoręcznie policzyć, ile BDP mogłoby kosztować Polskę (w wariancie pesymistycznym i optymistycznym).
  \par Koszty przyznania każdemu z 38 mln Polaków BDP w wysokości 1500 zł miesięcznie to 684 mld zł rocznie. Jest to kwota o ponad 350 mld złotych (a więc o ponad 100\%) większa niż wpływy do budżetu państwa za rok 2016. Wprowadzenie BDP w takim wariancie oznaczałoby wzrost wydatków sektora publicznego o ponad 100\%, jest więc nierealne.
  \par Nawet gdyby zabrać emerytom emerytury, zlikwidować pomoc społeczną (w tym 500+, zasiłki dla bezrobotnych i urzędy pracy) oraz przestać spłacać dług publiczny, koszty przyznania każdemu dorosłemu Polakowi (na oko ok. 30 mln ludzi) 1500 zł miesięcznie wyniosłyby w skali roku ok. 200 mld złotych, co przekłada się na wzrost wydatków publicznych o 29\%. Również ten wariant jest więc nierealny (obliczenia na podstawiwe mapy wydatków państwa za rok 2016).
  \item \textbf{Ekonomiści nie są przychylni BDP, bo są sługusami banków i korporacji}. Jest to stwierdzenie w pewnym sensie prawdziwe, ale znów raczej nie w takim jaki prof. Standing miał na myśli. Większość współczesnych ekonomistów to ekonomiści szkoły neoklasycznej, będący pod sporym wpływem keynesizmu, którzy są w mniejszym lub większym stopniu (raczej większym) zwolennikami interwencjonizmu państwowego. Ponieważ obecnie rynki są bardzo mocno uregulowane, mamy faktycznie do czynienia z pewną formą symbiozy państw i korporacji (dowodem tego stwierdzenia mogą być choćby przywołana przez Standinga akcja ratowania banków przez EBC i podobne sytuacje zza oceanu). Skoro więc ekonomiści są propaństwowi, to z konieczności są także prokorporacyjni i probankowi. Oczywiście nie dotyczy to wszystkich bez wyjątku i jest jedynie pewnym uogólnieniem.
  \par Mimo powyższego uważam, że niechęć ekonomistów do idei BDP wynika po prostu z rozsądku i trzeźwego spojrzenia na sprawę, podobnego do tego jakie zaprezentowałem analizując poprzedni punkt — zdają oni sobie sprawę z tego, że postulat wprowadzenia BDP jest niemożliwy do zrealizowania, przynajmniej obecnie, głównie ze względów finansowych.
  \item \textbf{Należy wprowadzić BDP}. Główna teza rozumowania prof. Standinga, na mocy powyższej analizy, jest nieprawdziwa. Przeciw niej świadczy kruchość podstaw filozoficznych i moralnych tego pomysłu oraz masa możliwości, których Standing nie wziął w swoich rozumowaniach pod uwagę. Jednak najmocniejszym argumentem za tym, że nie należy wprowadzać BDP, jest kwestia jego finansowania — na chwilę obecną sfinansowanie takiego projektu w Polsce jest niemożliwe.
\end{enumerate}

\end{document}